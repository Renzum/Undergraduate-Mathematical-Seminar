\documentclass[18pt]{beamer}

\usepackage{graphicx}
\usepackage{hyperref}
\usepackage{mathtools}
\DeclarePairedDelimiter{\ceil}{\lceil}{\rceil}
\usepackage{xcolor}
\graphicspath{{./images/}}

\usetheme{metropolis}
\setbeamertemplate{frame numbering}[fraction]
\useoutertheme{metropolis}
\useinnertheme{metropolis}
\usefonttheme{metropolis}
\usecolortheme{spruce}

\title{Undergraduate Seminar Presentation}
\subtitle{Number of Primes}
\author{Karen Arzumanyan}
\date{November 22, 2021}

\begin{document}
\metroset{block=fill}

\begin{frame}
    \maketitle
\end{frame}

\begin{frame}{The number of primes until $x$, the function $\pi(x)$}
    \begin{definition}
        The function $\pi(x)$ takes an integer, and returns the number of prime numbers that are less or equal to that integer.
        \pause{}
        \[\pi(x) = \sum\limits_{p \leq x} 1 = \#\{p \leq  x | p \text{ is prime.}\}\]
    \end{definition}
\end{frame}

\begin{frame}{Behavior of $\pi(x)$}
   The function $\pi(x)$ behaves roughly like $\frac{x}{\log{x}}$ as $x \rightarrow \infty$
   \pause{}
   \[\lim\limits_{x \rightarrow \infty} \pi(x)\frac{\log{x}}{x} = 1\]
   \pause{}
   This was proven by \textit{J. Hadamard} and \textit{C. de la Vallé Poussin} in 1896.\\
   \pause{}
   The proof is too complicated and is beyond the scope of this presentation.
\end{frame}

\begin{frame}{Chebyshev's Theorem}
    \begin{theorem}[Chebyshev]
    There exist constants $0 < c_1 < 1 < c_2$ such that \textit{for all} $x \geq 1$ we have that
    \[c_1 \frac{x}{\log{x}} \leq \pi(x) \leq c_2 \frac{x}{\log{x}}\]
    \end{theorem}
    \vspace{1em}
    \pause{}
    \textit{Pavnutii Lvovich Chebyshev} proved this theorem for $c_1 = 0.92$ and $c_2 = 1.11$.\\
    \pause{}
    \vspace{1em}
    We will prove the theorem for weaker bounds.
\end{frame}

\begin{frame}{Sum of logs}
   \begin{block}{Sum of Logs}
       We define a function \[\theta(x) = \sum\limits_{p \leq x} \log{p} \]
       and specify that $\theta(1) = 0$.
   \end{block} 
   \pause{}
   \vspace{0.5em}
   If we compare $\theta(x)$ and $\pi(x)$ we can see that since $\log{p} \leq \log{x}$
   \[\theta(x) = \sum\limits_{p \leq x} \log{p} \leq \sum \limits_{p \leq x} \log{x} = \pi(x) \log{x}\]
   \pause{}
   Therefore,
   \[\frac{\theta(x)}{x} \leq \frac{\pi(x)}{\big(\frac{x}{\log{x}}\big)}\]
\end{frame}

\begin{frame}{Small $\epsilon$}
    Suppose we have $0 < \epsilon < 1$, then
    \vspace{1em}
    \[\theta(x) = \sum\limits_{p \leq x} \log{p} = \sum\limits_{p \leq x^{1-\epsilon}} \log{p} + \sum\limits_{x ^{1 -\epsilon} < p \leq x} \log{p}\]
    \pause{}
    \[\geq (1-\epsilon)\Big(\pi(x)-\pi\big(x^{1-\epsilon}\big)\Big)\log{x}\]
    \pause{}
    \[\geq (1-\epsilon) \pi(x) \log{x} - \pi(x^{1-\epsilon}) \log{x}\]
    \pause{}
    \[\geq (1 - \epsilon) \pi(x)\log{x} - x^{1-\epsilon}\log{x}\]

\end{frame}

\begin{frame}{Small $\epsilon$}
    Now we have that
    \[\theta(x) \geq (1 - \epsilon) \pi(x)\log{x} - x^{1-\epsilon}\log{x}\]
    \pause{}
    If we divide both sides by $x$, we will get
    \[\frac{\theta(x)}{x} \geq (1-\epsilon)\frac{\pi(x)}{\big(\frac{x}{\log{x}}\big)} - \frac{\log{x}}{x^{\epsilon}}\]
    \pause{}
    \textbf{Remark:} Recall that we also had
  \[\frac{\theta(x)}{x} \leq \frac{\pi(x)}{\big(\frac{x}{\log{x}}\big)}\]
    Since the term $\frac{\log{x}}{x^{\epsilon} }$ vanishes as $x \rightarrow \infty$, we can go back and forth between asymptotic formulas of $\pi(x)$ and $\theta(x)$.

\end{frame}

\begin{frame}{Lemma for $\theta(x)$}
    \begin{Lemma}
        For all $x \geq 1$ we have,
        \[\theta(x) < (4\log{2})x\]
    \end{Lemma}
\end{frame}

\begin{frame}{Lemma          for $\theta(x)$}
    \begin{block}{Proof.}
        Let's consider the binomial coefficient
        \[{2n \choose n} = \frac{(2n)(2n-1)\ldots(n+1)}{n!} < 2^{2n} \]
        \pause{}
        where the inequality follows from
    \[2^{2n} = (1+1)^{2n} = \sum\limits_{k = 0} ^{2n} {2n \choose k}1^{2n-k}1^{k} > {2n \choose n}\]
        \pause{}
        We know that every prime $p \in (n, 2n]$ will appear in the numerator, but not in the denominator.\\
        \pause{}
        Therefore,
        \[\prod\limits_{n < p \leq 2n} p \leq {2n \choose n} < 2^{2n}\]
    \end{block}
\end{frame}

\begin{frame}{Proof for Lemma for $\theta(x)$}
    \begin{block}{Proof.}
        Hence, taking the $\log{}$ of both sides
        \[\sum\limits_{n < p \leq 2n} \log{p} < \log{2^{2n}}\]
        \pause{}
        That is,
        \[\theta(2n)-\theta(n) < (2\log{2})n\]
        \pause{}
        In particular, taking $n = 2^{k-1}$, we get that
        \[\theta(2^{k})-\theta(2^{k-1}) < (2\log{2})2^{k-1}\]
        \pause{}
    \end{block}
\end{frame}

\begin{frame}{Lemma          for $\theta(x)$}
    \begin{block}{Proof.}
        If we apply a telescopic sum,
        \[\sum\limits_{k=1} ^{m} \theta(2^{k}) - \theta(2^{k-1})\]
        \pause{}
        \[=\Big(\theta(2)-\theta(1)\Big) + \Big(\theta(4) - \theta(2)\Big) + \ldots + \Big(\theta(2^{m}) - \theta(2^{m-1})\Big)\]
        \pause{}
        \[= \theta(2^{m}) - \theta(1) = \theta(2^{m})\]
        \pause{}
        Thus, together with the inequality from before
        \[\theta(2^{m}) \leq (2\log{2}) \sum\limits_{k=1} ^{m} 2^{k-1} < (2\log{2})2^{m}\]
    \end{block}
\end{frame}

\begin{frame}{Lemma for $\theta(x)$}
    \begin{block}{Proof.}
       For an arbitrary $x \geq 1$, choose an integer $m$ such that $2^{m-1} \leq x < 2^{m}$.
       \pause{}
       Then,
       \[\theta(x) \leq \theta(2^{m}) < (2\log{2})2^{m} \leq (4\log{2})x\]
       \pause{}
       that is
       \[\theta(x) \leq (4\log{2})x\]
    \end{block}
\end{frame}

\begin{frame}{Upper bound of Chebyshev}
    \begin{block}{Proof.}
        We already know that,
            \[(4\log{2})x \geq \theta(x) \geq \sum\limits_{\sqrt{x} < p \leq x} \log{p}\]
        \pause{}
        \[\geq \log{(\sqrt{x})}(\pi(x) - \pi(\sqrt{x})) \geq \frac{1}{2}\log{(x)}(\pi(x)-\pi(\sqrt{x}))\]
        \pause{}
        \[\geq \frac{1}{2}\log{(x)}\pi(x)-\frac{1}{2}\log{(x)}\sqrt{x}\]
        where the last inequality follows from $\pi(\sqrt{x}) \leq \sqrt{x}$.\\
        Equivalently,
        \[\frac{1}{2}\log{(x)}\pi(x) \leq (4\log{2})x + \frac{1}{2}\log{(x)}\sqrt{x}\]
    \end{block}
\end{frame}

\begin{frame}{Upper bound of Chebyshev}
   \begin{block}{Proof.}
        Dividing both sides of the last inequality by $\frac{1}{2}\log{(x)}$, we get
        \[\pi(x) \leq (8\log{2})\frac{x}{\log{x}} + \sqrt{x} \leq (8\log{2}+2)\frac{x}{\log{x}}\]
        \pause{}
        where we used the simple inequality $\sqrt{x} < \frac{2x}{\log{x}}$ for all $x \geq 2$.
        \pause{}
    \pause{}
    Therefore
    \[c_2 = 8\log{2} + 2 \approx 4.41\]
   \end{block} 
\end{frame}

\begin{frame}{Lower bound of Chebyshev}
    \begin{block}{Proof of the Lower Bound}
        For the lower bound, let us consider the following binomial coefficient inequality
        \[{2n \choose n} = \prod\limits_{1 \leq k \leq n} \frac{k+n}{k} \geq 2^{n}\]
        \pause{}
        We now write
        \[{2n \choose n} = \prod\limits_{p < 2n} p^{\alpha_p} \geq 2^{n}\]
        for $\alpha_p > 0$
    \end{block}
\end{frame}

\begin{frame}{Lower bound of Chebyshev}
    \begin{lemma}
        The largest power of $p$ dividing $n!$ is given by
        \[\nu_p(n!) = \sum\limits_{k = 1} ^{\infty} \Bigr\lfloor{\frac{n}{p^{k}}}\Bigr\rfloor\]
    \end{lemma}
\end{frame}

\begin{frame}{Lower bound of Chebyshev}
    \begin{block}{Proof.}
        As ${2n \choose n} = \frac{2n!}{(n!)^{2}}$, then the power of $p$ in the decomposition of ${2n \choose n}$ can be calculated by subtracting twice the power of $p$ in the decomposition of $n!$ from the power of $p$ in the decomposition of $(2n)!$. Applying the previous lemma,
        \[\alpha_p = \sum\limits_{j = 1} ^{t_p} \Bigg(\Bigr\lfloor\frac{2n}{p^{j}}\Bigr\rfloor - 2\Bigr\lfloor\frac{n}{p^{j}}\Bigr\rfloor\Bigg) = \sum\limits_{j=1} ^{t_p} \Bigr\lfloor\frac{2n}{p_j}\Bigr\rfloor - 2 \sum\limits_{j = 1} ^{t_p} \Bigr\lfloor\frac{n}{p_j}\Bigr\rfloor\]
        where $t_p$ is the largest integer such that $p^{t_p} \leq 2n$.
        \newline
    \end{block}
\end{frame}

\begin{frame}{Lower bound of Chebyshev}
    \begin{block}{Proof.}
        Taking the $\log{}$ of both sides gives $t_p = \Bigr\lfloor\frac{\log{2n}}{\log{p}}\Bigr\rfloor$, implying
        \[n\log{2} \leq \sum\limits_{p < 2n} t_p \log{p} = \sum\limits_{p < 2n} \Bigr\lfloor\frac{\log{2n}}{\log{p}}\Bigr\rfloor \log{p}\]
        \pause{}
        Now, we need to separate the right hand side (RHS) into two separate sums and deal with them individually.
    \end{block}
\end{frame}

\begin{frame}{Lower bound of Chebyshev}
    \begin{block}{Proof.}
        We have,
        \[RHS = \sum\limits_{p < \sqrt{2n}} \Bigr\lfloor\frac{\log{2n}}{\log{p}} \Bigr\rfloor \log{p} + \sum\limits_{\sqrt{2n} \leq p \leq 2n} \log{p} \leq \sqrt{2n}\log{2n} + \theta(2n)\]
        \pause{}
        We have proven that for sufficiently large $n \geq 1$,
        \[\theta(2n) \geq n\log{2} - \sqrt{2n}\log{2n} \geq Cn\]
    \end{block}
\end{frame}

\begin{frame}{Lower bound of Chebyshev}
    \begin{block}{Proof.}
        For $x = 2n + 1$, we also have that
        \[\theta(x) \geq \theta(2n) \geq Cn \geq \frac{C}{4}x\]
        \pause{}
        It is now clear that
        \[\theta(x) = \sum\limits_{p \leq x} \log{p} \leq \pi(x) \log{x}\]
        \pause{}
        Therefore,
        \[\frac{C}{4}\frac{x}{\log{x}} \leq \frac{\theta(x)}{\log{x}} \leq \pi(x)\]
    \end{block}
\end{frame}

\begin{frame}{Work Cited}
    \begin{thebibliography}{1}
        \bibitem{M} Mallahi-Karai, K., \href{https://www.dropbox.com/s/msf4auc05s0d1kl/LN-complete_ENT20.pdf?dl=0}{Number Theory Complete Lecture Notes}.
    \end{thebibliography} 
\end{frame}
\end{document}
